\documentclass[twocolumn]{jsarticle}

\usepackage{listings, jlisting, color}
\usepackage[dvipdfmx]{graphicx}
\usepackage{pdfpages}
\usepackage{amsmath}
\usepackage{mathtools}
\usepackage{multirow}
\usepackage{color}
\usepackage{ulem}
\usepackage{here}
\usepackage{wrapfig}

\newcommand{\Add}[1]{\textcolor{red}{#1}}
\newcommand{\Erase}[1]{\textcolor{red}{\sout{\textcolor{black}{#1}}}}
\newcommand{\ctext}[1]{\raise0.2ex\hbox{\textcircled{\scriptsize{#1}}}}

\lstset{
  basicstyle={\ttfamily},
  identifierstyle={\small},
  commentstyle={\smallitshape},
  keywordstyle={\small\bfseries},
  ndkeywordstyle={\small},
  stringstyle={\small\ttfamily},
  frame={tb},
  breaklines=true,
  columns=[l]{fullflexible},
  numbers=left,
  xrightmargin=0zw,
  xleftmargin=3zw,
  numberstyle={\scriptsize},
  stepnumber=1,
  numbersep=1zw,
  lineskip=-0.5ex
}

\makeatletter % プリアンブルで定義開始

% 表示文字列を"図"から"Figure”へ,"表"から"Table”へ
\renewcommand{\figurename}{図}
\renewcommand{\tablename}{表}

% 図,表番号を"<章番号>.<図番号>” ,"<章番号>.<表番号>” へ
\renewcommand{\thefigure}{\thesection-\arabic{figure}}
\renewcommand{\thetable}{\thesection-\arabic{table}}

% 章が進むごとに図番号をリセットする
\@addtoreset{figure}{section}
\@addtoreset{table}{section}

\makeatother % プリアンブルで定義終了

\begin{document}
\twocolumn[
  \title{テーマ3 コンピュータネットワーク実験}
  \author{222C1021 今村優希}
  \maketitle
]

\section{はじめに}
ネットワークの発達によって、ユーザは自身の計算機を容易ネットワークに接続することができる。
その仕組を小規模なネットワークを作成し、ネットワークの状況を確認することで理解する。

\section{実験環境}
今回作成したネットワークは図である。

\begin{figure}[H]
  \begin{center}
    \includegraphics[scale=0.2]{graphic/1.png}
  \end{center}
  \caption{作成したネットワークのイメージ}
  \label{fig:1}
\end{figure}

クライアントPC1のIPアドレスを"192.168.1.18"、
クライアントPC2のIPアドレスを"192.168.3.21"に設定した。
PC1からPC2に接続する際に2つのルーターを経由する。

\section{実験方法}
今回の実験ではTcpdumpコマンドを用いてWebデータ受信時の挙動について確認を行った。
この実験を実施するために、まずはクライアントPC2をWebサーバーに設定した。
次にダウンロード用のファイルをサーバー側で用意した。
その後ログ取得のためにTcpdumpコマンドを実行した。

それと同時にクライアントでもTcpdumpコマンドを実行し、
wgetコマンドを用いてサーバー側で用意したファイルをダウンロードする。
ダウンロードが終了したらTcpdumpを終わらせてログを取得する。

以上の手順で実験を行った。

\section{実験結果}
クライアント側(IP: 192.186.1.18)で取得したログの一部を下記に記述する。\\
{\scriptsize \texttt{15:19:21.915462 IP (tos 0x0, ttl 62, id 43096, offset 0, flags [DF], proto TCP (6), length 1012)\\
  192.168.3.21.http > 192.168.1.18.49152: Flags [.], cksum 0x5e0e (correct), seq 310090316:310091276, ack 124, win 4113, options [nop,nop,TS val 394305168 ecr 20774075], length 960: HTTP\\
・・・\\
15:19:21.915562 IP (tos 0x0, ttl 64, id 57950, offset 0, flags [DF], proto TCP (6), length 52, bad cksum 0 (->d2ed)!)\\
  192.168.1.18.49152 > 192.168.3.21.http: Flags [.], cksum 0x859e (incorrect -> 0x5877), ack 310091276, win 5542, options [nop,nop,TS val 20774077 ecr 394305168], length 0\\
15:19:21.915595 IP (tos 0x0, ttl 62, id 8096, offset 0, flags [DF], proto TCP (6), length 1500)\\
  192.168.3.21.http > 192.168.1.18.49152: Flags [.], cksum 0x5866 (correct), seq 310091276:310092724, ack 124, win 4113, options [nop,nop,TS val 394305168 ecr 20774075], length 1448: HTTP\\
}}
前半はサーバーからクライアントへデータを送信した際のログ、後半はクライアントからサーバーへデータを送信した際のログである。
これらのログからACKの挙動に関して考察を行う。

\section{考察}
前半において、サーバーからクライアントにデータを送信しているログの中で、
シーケンス番号が"seq 310090316:310091276"となっているため310090316から310091276のパケットを送信していると考えられる。
そして、その時のACKが124であり、サーバーもACKを返していることが確認できた。
また、length 1448とあるので1448 byteのデータが送信されたと考えられる。

その後、クライアントがACKを返したあとに、再度サーバー側からクライアントへパケットを送信している。
その際、"seq 310091276:310092724"という記述から310091276から310092724のパケットを送信している。

ここで"310091276"という値に関して注目してみる。
この数字はサーバー側からシーケンス番号として送信され、クライアントがACKとして返し、最後に再度サーバー側からシーケンス番号として送られている。
これらの結果から、ACKはパケットを正式に受け取った場合に、パケットの中で一番最後の値、seq a:bで受け取った場合にbをACKとして返していると考えられる。

さらに、遅延ACKの挙動に関して、サーバー側からクライアントへパケットを数回連続で送信し、クライアントからサーバー側に数回連続でACKを返す挙動を確認することができた。
したがって、複数のパケットを一気に送る「スライディングウィンドウ」というものを使用していることが分かった。
具体的にログの記録を確認すると、ある時間帯では11回のパケットを送信し、ACKを返したのが7回であることから、2回に1回程度ACKを返していることが確認できた。
この動作を行うことで、ACKを返す回数が減るので、効率的にパケットを受信することができる事が考えられる。


\section{まとめ}
今回の実験では小規模なネットワークを構築し、データの送受信を通してネットワークの仕組みについて調べた。
実際にACKやシーケンス番号の経緯を確認し、TcpdumpコマンドにてWebデータ受信時の挙動について調べた。
\end{document}